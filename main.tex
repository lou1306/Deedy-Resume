% !TEX program = xelatex
% !BIB program = bibtex

%%%%%%%%%%%%%%%%%%%%%%%%%%%%%%%%%%%%%%%
% Deedy - One Page Two Column Resume
% LaTeX Template
% Version 1.1 (30/4/2014)
%
% Original author:
% Debarghya Das (http://debarghyadas.com)
%
% Original repository:
% https://github.com/deedydas/Deedy-Resume
%
% IMPORTANT: THIS TEMPLATE NEEDS TO BE COMPILED WITH XeLaTeX
%
% This template uses several fonts not included with Windows/Linux by
% default. If you get compilation errors saying a font is missing, find the line
% on which the font is used and either change it to a font included with your
% operating system or comment the line out to use the default font.
% 
%%%%%%%%%%%%%%%%%%%%%%%%%%%%%%%%%%%%%%
% 
% TODO:
% 1. Integrate biber/bibtex for article citation under publications.
% 2. Figure out a smoother way for the document to flow onto the next page.
% 3. Add styling information for a "Projects/Hacks" section.
% 4. Add location/address information
% 5. Merge OpenFont and MacFonts as a single sty with options.
% 
%%%%%%%%%%%%%%%%%%%%%%%%%%%%%%%%%%%%%%
%
% CHANGELOG:
% v1.1:
% 1. Fixed several compilation bugs with \renewcommand
% 2. Got Open-source fonts (Windows/Linux support)
% 3. Added Last Updated
% 4. Move Title styling into .sty
% 5. Commented .sty file.
%
%%%%%%%%%%%%%%%%%%%%%%%%%%%%%%%%%%%%%%%
%
% Known Issues:
% 1. Overflows onto second page if any column's contents are more than the
% vertical limit
% 2. Hacky space on the first bullet point on the second column.
%
%%%%%%%%%%%%%%%%%%%%%%%%%%%%%%%%%%%%%%

%% Mac-only
% \documentclass[a4paper]{deedy-resume}
% \newcommand*{\custombold}[1]{\textbf{#1}}

\documentclass[a4paper]{deedy-resume-openfont}

\usepackage[inline]{enumitem}

\newfontfamily{\FA}[Path = fonts/]{fontawesome-webfont}

\def\github{\color{gray}{\FA\symbol{"F092}}}
\def\facebook{\color{gray}{\FA\symbol{"F082}}}
\def\linkedin{\color{gray}{\FA\symbol{"F08C}}}
\def\phoneb{\color{gray}{\FA\symbol{"F098}}}
\def\home{\color{gray}{\FA\symbol{"F015}}}
\def\mail{\color{gray}{\FA\symbol{"F0E0}}}
\def\globe{\color{gray}{\FA\symbol{"F0AC}}}

\makeatletter
\renewcommand\@biblabel[1]{\textbullet}
\makeatother
\usepackage{microtype}
\usepackage[numbers]{natbib}
\bibliographystyle{unsrtnat}
\renewcommand{\bibsection}{}

\hypersetup{
    colorlinks,
    citecolor=black,
    filecolor=black,
    linkcolor=black,
    urlcolor=black
}

\begin{document}

% \lastupdated

\namesection{Luca}{Di Stefano}{
%Gran Sasso Science Institute • Via Michele Iacobucci, 2 • 67100 L'Aquila, Italy\\
\begin{itemize*}[itemjoin = \hspace{1em}]
    \item[\mail] \href{mailto:luca.distefano@gssi.it}{luca.distefano@gssi.it} 
    \item[\phoneb] ...
    \item[\globe] \urlstyle{same}\href{http://lou1306.github.io}{lou1306.github.io}
    \item[\linkedin] \href{https://www.linkedin.com/in/lou1306}{linkedin.com/in/lou1306}
\end{itemize*}
}

\begin{minipage}[t]{0.3\textwidth} 

\section{WORK EXPERIENCE} 

\runsubsection{Thales Alenia Space}
\descript{Workshop Planning Intern}
\location{June 2015 – Feb 2016 • L'Aquila, IT}
\vspace{\topsep} % Hacky fix for awkward extra vertical space
Maintenance of legacy traceability systems; development of in-house client applications (SQL Server, VBA).
\sectionsep

\runsubsection{Aubay Research \& Technologies}
\descript{Intern}
\location{Dec 2013 – Feb 2014\ • Carsoli, IT}
\vspace{\topsep} % Hacky fix for awkward extra vertical space
Bachelor's thesis project.
Design of a digital publishing platform with a cross-platform mobile client (C\#, ASP.NET, Windows Communication Foundation, Java, Xamarin).
\sectionsep

\section{VOLUNTEERING}

\subsection{Associazione Italiana Guide e Scouts d'Europa Cattolici}
\location{January 2015 – Present time}
\vspace{\topsep} % Hacky fix for awkward extra vertical space
Development and deployment of web applications
(C\#, ASP.Net, SQL Server, PHP, JavaScript, Azure).
\sectionsep

\section{PROGRAMMING}
\location{Proficient:}
Python • C\# • F\# • Java • \LaTeX\ \\
\location{Competent:}
C • C++ • Javascript • T-SQL\\ Matlab • Bash • Erlang \\
\location{Familiar:}
OCaml • PHP • Prolog • VHDL
\sectionsep

\section{CERTIFICATIONS} 
\vspace{\topsep} % Hacky fix for awkward extra vertical space
\begin{tightemize}
\item[2015]    Cisco CCNA Routing and Switching: Introduction to Networks\\
\item[2010]    Cambridge First Certificate in English (FCE). Pass with merit\\
\end{tightemize}
\sectionsep

\end{minipage} 
\hfill
\begin{minipage}[t]{0.67\textwidth} 

\section[Publications]{PUBLICATIONS \small{(DBLP: \href{https://dblp.org/pid/215/9758}{\custombold{215/9758}}, ORCID iD: \href{https://orcid.org/0000-0003-1922-3151}{\custombold{0000-0003-1922-3151}})}}
\vspace{\topsep} % Hacky fix for awkward extra vertical space

\nocite{*}
\bibliography{publications}

% \begin{tightemize}
%     \item R. De Nicola, L. Di Stefano, and O. Inverso, ``Multi-Agent Systems with Virtual Stigmergy,'' to appear in Science of Computer Programming.
%     \item R. De Nicola, L. Di Stefano, and O. Inverso, ``Toward Formal Models and Languages for Verifiable Multi-Robot Systems,'' Front. Robot. AI, vol. 5, 2018.
%     \item R. De Nicola, L. Di Stefano, and O. Inverso, ``Multi-Agent Systems with Virtual Stigmergy,'' in STAF 2018 Collocated Workshops, Revised Selected Papers,
%     Toulouse, France, 2018, pp. 351-366. 
%     \item L. Di Stefano, E. Clementini, and E. Stagnini, ``Reactive Obstacle Avoidance for Multicopter UAVs via Evaluation of Depth Maps,'' in COSIT 2017 Workshops and Posters, L’Aquila, Italy, 2017, pp. 41–43.
% \end{tightemize}

\sectionsep

%%%%%%%%%%%%%%%%%%%%%%%%%%%%%%%%%%%%%%
%     EXPERIENCE
%%%%%%%%%%%%%%%%%%%%%%%%%%%%%%%%%%%%%%

\section{EDUCATION}

\runsubsection{INRIA}
\descript{Visiting PhD student}
\location{March – July 2019 • Grenoble, FR}
\begin{tightemize}
    \item Encoding of multi-agent systems in the LNT specification language
    \item Verification of multi-agent systems through model checking (\href{https://cadp.inria.fr/}{\custombold{CADP}})
\end{tightemize}
\sectionsep

\subsection{1st VMCAI Winter School}
\location{January 2019 • Lisbon, PT}
\begin{tightemize}
    \item Computing with SAT oracles
    \item Abstract interpretation
    \item Modeling and verification of distributed protocols
\end{tightemize}
\sectionsep

\runsubsection{Gran Sasso Science Institute}
\descript{PhD in Computer Science}
\location{October 2016 – Present time • L'Aquila, IT}
\begin{tightemize}
\item Languages and process algebras for the specification and verification of reactive systems
\item Semantics of programming languages
\item Model checking and software verification
\end{tightemize}
Other activities: student representative in the academic senate
\sectionsep

\subsection{University of L'Aquila}
\descript{MS in Computer Science and Systems Engineering, cum laude}
\location{Oct 2016 • L'Aquila, IT}
\begin{tightemize}
\item Control Theory
\item Communication networks
\item Spatial information theory
\item Embedded systems
\end{tightemize}
\descript{BS in Computer Science and Systems Engineering}
\location{Mar 2014 • L'Aquila, IT}
\sectionsep

\section{SOFTWARE PROJECTS}
\location{\hspace{1em}}

\begin{tightemize}
\item
A tool to verify multi-agent systems with stigmergic interaction.
The tool transforms a multi-agent system specification into a sequential
imperative program, which can be verified with any off-the-shelf tool for
program analysis in the target language (Python, F\#).
Code partially available on \href{https://github.com/labs-lang/sliver}{\custombold{GitHub}}.

\item 
An open-source toolchain to synthetize garbled circuits from Python specifications (Python, Yosys, TinyGarble).
Assignment for a Cryptography and Security course at GSSI.
Code available on \href{https://github.com/lou1306/gssi/blob/master/2pc/}{\custombold{GitHub}}.
\item
A computer vision algorithm
for goal-oriented obstacle avoidance (Python, OpenCV, V-REP).
Master's thesis project; presented as a poster at COSIT2017. 
Code available on \href{https://github.com/lou1306/localpathplanner}{\custombold{GitHub}}.
\end{tightemize}

\end{minipage} 
\end{document}