\documentclass[11pt,a4paper]{moderncv}
\usepackage[T1]{fontenc}
\usepackage{microtype}
% \usepackage{times}
% \usepackage[top=5em, bottom=10em, hscale=0.63]{geometry}
\usepackage[a4paper, hscale=0.67]{geometry}
% \recomputelengths

\usepackage[bottom]{footmisc}

\moderncvstyle{banking}
\moderncvcolor{black}
\nopagenumbers{}

%% Comment this to remove references
% \newcommand*{\REFERENCES}{}

% Publications %%%%%%%%%%%%%%%%%%%%%%%%%%%%%%%%%%%%%%%%%%%%%%%%%%%%%%%%%%%%%%%%%
%% Set base style for biblatex-publist (do not use)
% \newcommand*\publistbasestyle{mla}

% Comment this to remove markers and other stuff related to highlighted publications
% \newcommand{\HIGHLIGHT}{}
\usepackage[%
    doi=false,bibstyle=publist,plnumbering=none,plauthorhandling=highlight,%
    hlyear=false,linktitles=doi,maxbibnames=99,citestyle=authoryear,
    giveninits=true,nameorder=given-family,uniquename=init
]{biblatex}
\ifx\HIGHLIGHT\undefined\else%
\renewbibmacro*{begentry}{%
\ifkeyword{highlight}{\makebox[0pt][r]{$\star$\addspace}}{}}
\fi
% \DefineBibliographyStrings{english}{%
%   toappear = {To appear},
% }

% \renewbibmacro*{begentry}{\ifkeyword{highlight}{\makebox[0pt][r]{$\star$\addspace}}{}}
% \plauthorname[Luca]{Di Stefano}
% \renewcommand*{\plauthorhl}[1]{#1}
\DeclareFieldInputHandler{extradate}{\def\NewValue{}}
\addbibresource{../publications.bib}
\defbibcheck{isjournal}{\ifentrytype{article}{}{\skipentry}}
\defbibcheck{isconference}{\ifentrytype{inproceedings}{}{\skipentry}}
\defbibcheck{ischapter}{\ifentrytype{inbook}{}{\skipentry}}
\defbibcheck{isreport}{\ifentrytype{report}{}{\ifkeyword{preprint}{}{\skipentry}}}
\defbibcheck{ismastersthesis}{\ifkeyword{mastersthesis}{}{\skipentry}}
\defbibheading{journals}{\subsection*{Journal articles}}
\defbibheading{conferences}{\subsection*{Peer-reviewed conference and workshop papers}}
\defbibheading{chapters}{\subsection*{Book chapters}}
\defbibheading{reports}{\subsection*{Preprints and technical reports}}
\defbibheading{masters}{\subsection*{Supervised Master's students}}
%%%%%%%%%%%%%%%%%%%%%%%%%%%%%%%%%%%%%%%%%%%%%%%%%%%%%%%%%%%%%%%%%%%%%%%%%%%%%%%%
\usepackage{datetime}

\newcommand*{\SE}{\ifShort{SE}\ifLong{Sweden}}
\newcommand*{\FR}{\ifShort{FR}\ifLong{France}}
\newcommand*{\IT}{\ifShort{IT}\ifLong{Italy}}
\newcommand*{\PT}{\ifShort{PT}\ifLong{Portugal}}
\newcommand*{\MONTH}[1]{\ifShort{\shortmonthname[#1]}\ifLong{\monthname[#1]}}

\author{Luca Di Stefano}
\date{\today}

\firstname{Luca}
\familyname{Di Stefano}
\extrainfo{Curriculum Vitae} 
\newdateformat{monthyear}{\monthname[\THEMONTH] \THEYEAR}
\quote{\normalfont \monthyear\today}
\usdate

\newcommand{\comboJob}[5]{%
\cventry{#3}{#2}{#1}{#4}{}{#5}%
}
\newcommand{\comboJobZero}[5]{%
\cventry[0pt]{#3}{#2}{#1}{#4}{}{#5}%
}

\newcommand*{\custombold}[1]{\textbf{#1}}
\newcommand*{\ifShort}[1]{}
\newcommand*{\ifLong}[1]{#1}


\newenvironment{tightemize}{\begin{itemize}\setlength\itemsep{0.2em}}{\end{itemize}}

\begin{document}

\makeatletter
\bfseries
\noindent {\Large\@author}\vspace*{0.4em}

\normalfont
\noindent {Curriculum vitae}

\noindent{\monthyear\today}
\makeatother

\section*{Personal information}
\begin{description}
    \item[E-mail:] \href{mailto:luca.di.stefano@tuwien.ac.at}{luca.di.stefano@tuwien.ac.at} 
    \item[Work address:] Treitlstraße 3, 1040, Vienna, Austria
    \item[Web page:] \url{https://www.lucadistefano.eu}
    \item[ORCID:] \url{https://orcid.org/0000-0003-1922-3151}
    \item[Scopus ID:] \href{https://www.scopus.com/authid/detail.uri?authorId=57205057932}{57205057932}
\end{description}

\section{Research interests}
My research mainly concerns the formal modelling and analysis of agent-based
models of complex collective systems.
On one hand, this entails the development of formally
defined high-level languages to concisely describe the features of individual
agents; on the other hand, it requires applying and improving state-of-the-art
verification techniques to check the collective behaviour of the resulting system.

Keywords associated with my interests include:

\cvline{Modelling}{%
Agent-based modelling,
Attribute-based communication,
Collective adaptive systems,
Domain-specific languages,
Multi-agent systems, 
Process algebras, 
Stigmergic interaction,
Structural operational semantics,
Temporal logics.
}

\cvline{Analysis}{%
Bounded model checking,
Explicit-state and symbolic model checking,
Software verification,
Static analysis.
}

\section*{Education}
\comboJob
{Gran Sasso Science Institute}
{PhD in Computer Science}
{\MONTH{11} 2016 – \MONTH{10} 2020}
{L'Aquila, \IT}
{\ifLong{
\cvline{Thesis}{\emph{Modelling and Verification of Multi-Agent Systems via Sequential Emulation}}
\cvline{Advisors}{Rocco De Nicola, Omar Inverso}
\cvline{URL}{\href{http://hdl.handle.net/20.500.12571/10181}{hdl.handle.net/20.500.12571/10181}}
\cvline{Other activities}{Student representative in the academic senate (2018--2020).}
}
\ifShort{\begin{tightemize}
    \item Languages and process algebras for the specification of reactive systems;
    % \item Semantics of programming languages
    \item Model checking and software verification.
    \item Other activities: student representative in the academic senate (2018--2020).
\end{tightemize}}
}

\comboJob
{University of L'Aquila}
{MSc in Computer Science and Systems Engineering\ifLong{\footnotemark[1]}}
{\MONTH{3} 2014 – \MONTH{10} 2016}
{L'Aquila, \IT}
{\ifLong{
    \cvline{Thesis}{\emph{Design of a reactive system for autonomous UAV navigation in unknown environments}\footnotemark[2]}
    \cvline{Advisors}{Eliseo Clementini, Enrico Stagnini}
    \cvline{Final grade}{110/110, \emph{cum laude}}
}
%
%
\ifShort{%
Final grade: 110/110, cum laude.
\begin{tightemize}
\item Control theory;
\item Communication networks;
\item Spatial information theory;
\item Embedded systems.
\end{tightemize}
}}

\ifLong{%
\footnotetext[1]{Official degree name: \emph{Ingegneria Informatica e Automatica}.}
\footnotetext[2]{Original title: \emph{Progettazione di un sistema reattivo per la navigazione autonoma di un drone in ambienti sconosciuti}.}
}


\section*{Academic Career}

\comboJob{\ifShort{INRIA}\ifLong{CONVECS, Inria/LIG}}{Post-doctoral researcher \ifShort{• CONVECS}}{\MONTH{11} 2020 – Present time}{Grenoble, \FR}{
    \ifLong{%
    \cvlistitem{Model checking temporal properties of collective adaptive systems;}
    \cvlistitem{Compositional verification of multi-agent systems.}}
}

\comboJob{IMT Lucca}{Grant holder}{\MONTH{12} 2019 – \MONTH{10} 2020}{Lucca, \IT}{
    \ifShort{
    \begin{tightemize}
        \item[] SysMA research unit.
        \item[] Research grant on ``Verification of Emerging Properties in Collective Adaptive Systems''.
    \end{tightemize}}%
    \ifLong{
         Research grant on ``Verification of Emerging Properties in Collective Adaptive Systems'',
         awarded by the SysMA research unit.
    }
}

\comboJob%
{\ifShort{INRIA}\ifLong{CONVECS, Inria/LIG}}%
{Visiting PhD student}%
{\MONTH{3} – \MONTH{7} 2019}%
{Grenoble, \FR}%
{\ifLong{
    \cvlistitem{Encoding of multi-agent systems in the LNT specification language;}
    \cvlistitem{Verification of multi-agent systems through model checking.}
}}

\section*{Teaching}
\comboJob
{Chalmers}
{BSc course in Principle of Concurrent Programming}
{\MONTH{1} – \MONTH{3} 2023}
{Gothenburg, \SE}
{
\ifLong{%
% \cvline{
Teaching assistant (approx. 80 hours).
The course covered both shared-memory and message-passing
concurrency, using Java and Erlang as reference languages.
I provided assistance in lab sessions, grading of assignments and exams, and
Erlang coding tutorials.
%
% \cvlistitem{Concurrent systems. Communicating automata. Behavioural equivalences.}
% \cvlistitem{Real-time systems. Timed automata.}
% \cvlistitem{Process algebras: CCS, LNT.}
% \cvlistitem{Modal and temporal logics: HML, $\mu$-calculus, MCL. Model checking.}
% \cvlistitem{Model-based testing. Input-output conformance. Test case generation.}%
}
\ifShort{%
\begin{tightemize}
\item[] Shared-memory (Java) and message-passing (Erlang) concurrency
\item[] Approx. 80 hours of teaching assistance (lab sessions, grading, Erlang tutorial)
}
}

\comboJobZero
{Polytech Paris-Saclay}
{MSc course in Modelling and Verification}
{\MONTH{3} – \MONTH{4} 2022}
{Paris, \FR}
{}
\comboJob
{}
{}
{\MONTH{4} – \MONTH{5} 2021}
{}
{
\ifLong{%
% \cvline{
36-hour course for Master students in Computer Science Engineering,
\textit{filière apprentissage}.
Held remotely (2021) and in person (2022) as a supply teacher (\emph{intervenant vacataire}).
The course focused on the following topics:
modelling of concurrent systems through communicating automata (labelled
transition systems); behavioural equivalences; process algebras
(CCS, LNT); modal and temporal logics (HML, $\mu$-calculus, MCL);
model checking; modelling of real-time systems through timed automata;
model-based testing.
Tools such as CADP and UPPAAL were showcased in lab sessions.
%
%
% \cvlistitem{Concurrent systems. Communicating automata. Behavioural equivalences.}
% \cvlistitem{Real-time systems. Timed automata.}
% \cvlistitem{Process algebras: CCS, LNT.}
% \cvlistitem{Modal and temporal logics: HML, $\mu$-calculus, MCL. Model checking.}
% \cvlistitem{Model-based testing. Input-output conformance. Test case generation.}%
}
%
\ifShort{%
\begin{tightemize}
    \item[] 36-hour course for Master students in Computer Science Engineering
    (\textit{filière apprentissage}). Held remotely as a supply teacher (\emph{intervenant vacataire}).
    \item Concurrent, reactive, real-time systems. Communicating automata. Timed automata. 
    \item Behavioral equivalences of communicating automata.
    \item Process algebras: CCS, LNT. Modal and temporal logics: HML, $\mu$-calculus, MCL.
    \item Model-based testing. Input-output conformance. Test case generation.
    \item Lab sessions: 
    Uppaal (verification of timed automata);
    CADP (verification of concurrent systems);
    Testor (test case generation).
\end{tightemize}%
}
}



\printbibliography[heading=masters, check=ismastersthesis]

% \subsection*{Supervised students}
% \cvline{Master students}{%
% Love Lyckaro, Chalmers, 2023.
% \emph{Evaluating in-memory caching strategies for distributed Web services}.%
% }

\section*{Other academic activities}

\subsection{Reviewing activity}

\cvline{Conferences}{%
SEFM~2019,
TASE~2019,
%%%%%%%%%%%
AAMAS~2021,
FM~2021,
ICSOFT~2021,
ISoLA~2021,
%%%%%%%%%%%%
FORTE~2022,
ISoLA~2022,
%%%%%%%%%%%%
CSL~2023,
iFM~2023,
TACAS~2023,
%%%%%%%%%%%%
EMSOFT~2024,
FMICS~2024,
iFM~2024,
ISoLA~2024.
}

\cvline{Artifact evaluations}{%
COORDINATION~2023.
}

\cvline{Journals}{%
Logical Methods in Computer Science (2023),
Science of Computer Programming (2022, 2024),
Software Tools for Technology Transfer (2023, 2024).
}

\subsection*{Program Committees}
FMICS~2024, NSAD~2024, ASQAP~2025.

\subsection*{Workshop Chair}
FTfJP~2024.

% \vspace*{\itemsep}
% \cvlistitem{International Conference on Autonomous Agents and Multiagent Systems (AAMAS)}
% \cvlistitem{International Symposium on Formal Methods (FM)}
% \cvlistitem{International Conference on Formal Techniques for Distributed Objects, Components, and Systems (FORTE)}
% \cvlistitem{International Symposium On Leveraging Applications of Formal Methods, Verification and Validation (ISoLA)}
% \cvlistitem{International Conference on Software Technologies (ICSOFT)}
% \cvlistitem{Science of Computer Programming, Elsevier}
% \cvlistitem{International Conference on Software Engineering and Formal Methods (SEFM)}
% \cvlistitem{Theoretical Aspects of Software Engineering Conference (TASE)}

\subsection*{Participation to PhD schools}
\comboJob
{1st VMCAI Winter School}
{\ifShort{Attendee}}
{\MONTH{1} 2019}
{Lisbon, \PT}
{\ifLong{
\begin{tightemize}
    \item Computing with SAT oracles;
    \item Abstract interpretation;
    \item Modeling and verification of distributed protocols.
\end{tightemize}
}}


\subsection*{Invited presentations}

\cvlistitem{\emph{Compositional Verification of Stigmergic Collective Systems.}
Invited presentation to the CONVECS research team (Inria/LIG).
Grenoble, France, 16 May 2023.
}

\cvlistitem{\emph{Verifying collective adaptive systems by emulation.}
Remote presentation to the Formal Methods unit of the Computer Science and
Engineering department (University of Gothenburg and Chalmers).
Gothenburg, Sweden, 10 March 2022.
}

\cvlistitem{\emph{Multi-agent (smart) systems with virtual stigmergies.}
Kickoff meeting of Italian national research project (PRIN)
\emph{IT MATTERS: Methods and Tools for Trustworthy Smart Systems.}
Pisa, Italy, 14 October 2019.
}
\cvlistitem{\emph{Multi-agent systems with virtual stigmergies.}
Invited presentation at IMT School for Advanced Studies.
Lucca, Italy, 3 July 2018.
}

\section*{Skills}
\cvline{Technical skills}{%
    My experience in programming spans more than a decade.
    As of now my languages of choice are mainly F\# and Python, but I have
    worked with a number of popular languages across the years
    (e.g., C, C++, C\#, Erlang, Go, Java, Prolog).
    I am familiar with software verification tools (CBMC, ESBMC, Infer),
    model checking software (CADP, NuXmv, SPIN),
    and SAT/SMT solvers (Glucose, MathSat, Minisat, Z3).
    I have some knowledge of control theory, computer architectures,
    assembly languages, and hardware description languages (mainly VHDL).
    I am also familiar with version control systems such as Git and
    Subversion.}
\vspace*{\itemsep}    
\cvline{Language skills}{
Italian is my first language.
I am fluent in English (CEFR level C1--C2)
and have some knowledge of French (CEFR level A2--B1).
I have basic Swedish skills (CEFR level A1).
}
% \cvline{Soft skills}{
% Having been the main contributor to multiple peer-reviewed scientific articles,
% I have acquired good academic writing skills, as well as an ability
% to work in groups, meet deadlines, and present my ideas both to my teammates
% and to external audiences.
% }

\section*{Publications}
\ifx\HIGHLIGHT\undefined\else%
Most relevant publications are highlighted by a $\star$.
\fi

\nocite{*}
\printbibliography[heading=journals, check=isjournal]
\printbibliography[heading=conferences, check=isconference]
\printbibliography[heading=chapters, check=ischapter]
\printbibliography[heading=reports, check=isreport]

\ifx\HIGHLIGHT\undefined\else%
\subsection*{Highlighted publications}
I was the main contributor to all the highlighted publications, except
(\cite{DBLP:journals/fmsd/StefanoL24}), where I mainly worked on the experimental
evaluation and on building a replication package to reproduce the results
presented in the article.
\fi

\subsection{Citation metrics}

As reported by Scopus on \today.

\cvline{}{} % For spacing
\cvline{Citations}{68}

\cvline{\emph{h}-index}{5}

% \section*{Received research grants}

% \cvlistitem{%
%     ``Verification of Emerging Properties in Collective Adaptive Systems''.
%     Short-term grant, December 2019 -- October 2020, IMT Lucca, Italy (see page~\pageref{grant:sysma}).}

    % \clearpage

\ifx\REFERENCES\undefined\else%
\section*{References}

For references, please contact: 

\vspace*{\itemsep}
\cvlistitem{%
    Rocco De Nicola,
    Professor of Computer Science,
    IMT School for Advanced Studies, Lucca, Italy.
    E-mail:~\href{mailto:rocco.denicola@imtlucca.it}{rocco.denicola@imtlucca.it}}
\cvlistitem{%
    Omar Inverso,
    Assistant Professor,
    Gran Sasso Science Institute, L'Aquila, Italy.
    E-mail:~\href{mailto:omar.inverso@gssi.it}{omar.inverso@gssi.it}}
% \cvlistitem{%
%     Frédéric Lang,
%     Research Fellow,
%     % Senior researcher,
%     Inria and LIG, Montbonnot Saint-Martin, Grenoble, France.
%     \mbox{E-mail:}~\href{mailto:frederic.lang@inria.fr}{frederic.lang@inria.fr}}

\cvlistitem{%
    Radu Mateescu,
    Research Director,
    % Senior researcher,
    Inria and LIG, Montbonnot Saint-Martin, Grenoble, France.
    \mbox{E-mail:}~\href{mailto:radu.mateescu@inria.fr}{radu.mateescu@inria.fr}}

\cvlistitem{%
    Nir Piterman,
    Full Professor,
    % Senior researcher,
    University of Gothenburg, Sweden.
    \mbox{E-mail:}~\href{mailto:piterman@chalmers.se}{piterman@chalmers.se}}

\vspace*{\itemsep}
Rocco De Nicola and Omar Inverso supervised my PhD thesis from 2017 to 2020.
% Frédéric Lang is a permanent member of the CONVECS research team
% (led by Radu Mateescu), where I worked as a post-doctoral researcher from 2020
% to 2022.
Radu Mateescu leads the CONVECS research team, where I worked as a post-doctoral researcher from 2020
to 2022.
Nir Piterman is the leader of my current research team in Gothenburg.
\fi

\end{document}
