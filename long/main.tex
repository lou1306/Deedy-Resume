\documentclass[11pt,a4paper]{moderncv}
\usepackage{microtype}
% \usepackage{times}
\usepackage[hscale=0.7,vscale=0.9]{geometry}
\recomputelengths

\usepackage[bottom]{footmisc}

\moderncvstyle{banking}
\moderncvcolor{black}
\nopagenumbers{}

% Publications %%%%%%%%%%%%%%%%%%%%%%%%%%%%%%%%%%%%%%%%%%%%%%%%%%%%%%%%%%%%%%%%%
\usepackage[doi=false,bibstyle=publist,plnumbered=false,plauthorhandling=highlight,boldyear=false,nameorder=given-family,linktitledoi=true]{biblatex}
\plauthorname[Luca]{Di Stefano}
\renewcommand*{\plauthorhl}[1]{#1}
\DeclareFieldInputHandler{extradate}{\def\NewValue{}}
\addbibresource{../publications.bib}
\defbibcheck{isjournal}{\ifentrytype{article}{}{\skipentry}}
\defbibcheck{isconference}{\ifentrytype{inproceedings}{}{\skipentry}}
\defbibheading{journals}{\subsection*{Journal articles}}
\defbibheading{conferences}{\subsection*{Peer-reviewed conference papers}}
%%%%%%%%%%%%%%%%%%%%%%%%%%%%%%%%%%%%%%%%%%%%%%%%%%%%%%%%%%%%%%%%%%%%%%%%%%%%%%%%
\usepackage{datetime}

\newcommand*{\SE}{\ifShort{SE}\ifLong{Sweden}}
\newcommand*{\FR}{\ifShort{FR}\ifLong{France}}
\newcommand*{\IT}{\ifShort{IT}\ifLong{Italy}}
\newcommand*{\PT}{\ifShort{PT}\ifLong{Portugal}}
\newcommand*{\MONTH}[1]{\ifShort{\shortmonthname[#1]}\ifLong{\monthname[#1]}}

\author{Luca Di Stefano}
\date{\today}

\firstname{Luca}
\familyname{Di Stefano}
\extrainfo{Curriculum Vitae} 
\newdateformat{monthyear}{\monthname[\THEMONTH] \THEYEAR}
\quote{\normalfont \monthyear\today}
\usdate

\newcommand{\comboJob}[5]{%
\cventry{#3}{#2}{#1}{#4}{}{#5}%
}

\newcommand*{\custombold}[1]{\textbf{#1}}
\newcommand*{\ifShort}[1]{}
\newcommand*{\ifLong}[1]{#1}

\newenvironment{tightemize}{\begin{itemize}\setlength\itemsep{0.2em}}{\end{itemize}}

\begin{document}

\makeatletter
\bfseries
\noindent {\Large\@author}\vspace*{0.4em}

\normalfont
\noindent {Curriculum vitae}

\noindent{\monthyear\today}
\makeatother

\section*{Personal information}
\begin{description}
    \item[E-mail:] \href{mailto:luca.di-stefano@inria.fr}{luca.di-stefano@inria.fr} 
    \item[Work address:] 655 avenue de l'Europe, 38334 Montbonnot, France
    \item[Web page:] \href{https://convecs.inria.fr/people/Luca.Di-Stefano/}{convecs.inria.fr/people/Luca.Di-Stefano/} 
\end{description}

\section{Research interests}
My research mainly concerns the formal modelling and analysis of multi-agent and
collective systems. This entails, on one hand, the development of formally
defined high-level languages to concisely describe the features of individual
agents, and on the other hand the application of state-of-the-art
verification techniques to check the collective behaviour of the system.
%
Keywords associated with my interests include:

\cvline{Modelling}{%
Multi-agent systems, collective adaptive systems,
Agent-based modelling, Stigmergic interaction,
Process algebras, Structural operational semantics,
Attribute-based communication, Domain-specific languages.
}
\cvline{Analysis}{%
Explicit-state and symbolic model checking,
Bounded model checking, Temporal logics, Software verification.
}


\section*{Education}
\comboJob
{Gran Sasso Science Institute}
{PhD in Computer Science}
{\MONTH{11} 2016 – \MONTH{10} 2020}
{L'Aquila, \IT}
{\ifLong{
\cvline{Thesis}{\emph{Modelling and Verification of Multi-Agent Systems via Sequential Emulation}}
\cvline{Advisors}{Rocco De Nicola, Omar Inverso}
\cvline{URL}{\href{http://hdl.handle.net/20.500.12571/10181}{hdl.handle.net/20.500.12571/10181}}
\cvline{Other activities}{Student representative in the academic senate (2018--2020).}
}
\ifShort{\begin{tightemize}
    \item Languages and process algebras for the specification of reactive systems;
    % \item Semantics of programming languages
    \item Model checking and software verification.
    \item Other activities: student representative in the academic senate (2018--2020).
\end{tightemize}}
}

\comboJob
{University of L'Aquila}
{MSc in Computer Science and Systems Engineering\ifLong{\footnotemark[1]}}
{\MONTH{3} 2014 – \MONTH{10} 2016}
{L'Aquila, \IT}
{\ifLong{
    \cvline{Thesis}{\emph{Design of a reactive system for autonomous UAV navigation in unknown environments}\footnotemark[2]}
    \cvline{Advisors}{Eliseo Clementini, Enrico Stagnini}
    \cvline{Final grade}{110/110, \emph{cum laude}}
}
%
%
\ifShort{%
Final grade: 110/110, cum laude.
\begin{tightemize}
\item Control theory;
\item Communication networks;
\item Spatial information theory;
\item Embedded systems.
\end{tightemize}
}}

\ifLong{%
\footnotetext[1]{Official degree name: \emph{Ingegneria Informatica e Automatica}.}
\footnotetext[2]{Original title: \emph{Progettazione di un sistema reattivo per la navigazione autonoma di un drone in ambienti sconosciuti}.}
}

\section*{Experience}

\comboJob{\ifShort{INRIA}\ifLong{CONVECS, Inria/LIG}}{Post-doctoral researcher \ifShort{• CONVECS}}{\MONTH{11} 2020 – Present time}{Grenoble, \FR}{
    \ifLong{%
    \cvlistitem{Model checking temporal properties of collective adaptive systems;}
    \cvlistitem{Compositional verification of multi-agent systems.}}
}

\comboJob{IMT Lucca}{Grant holder}{\MONTH{12} 2019 – \MONTH{10} 2020}{Lucca, \IT}{
    \ifShort{
    \begin{tightemize}
        \item[] SysMA research unit.
        \item[] Research grant on ``Verification of Emerging Properties in Collective Adaptive Systems''.
    \end{tightemize}}%
    \ifLong{
         Research grant on ``Verification of Emerging Properties in Collective Adaptive Systems'',
         awarded by the SysMA research unit.
    }
}

\comboJob%
{\ifShort{INRIA}\ifLong{CONVECS, Inria/LIG}}%
{Visiting PhD student}%
{\MONTH{3} – \MONTH{7} 2019}%
{Grenoble, \FR}%
{\ifLong{
    \cvlistitem{Encoding of multi-agent systems in the LNT specification language;}
    \cvlistitem{Verification of multi-agent systems through model checking.}
}}

\section*{Teaching}
\comboJob
{Chalmers}
{BSc course in Principle of Concurrent Programming}
{\MONTH{1} – \MONTH{3} 2023}
{Gothenburg, \SE}
{
\ifLong{%
% \cvline{
Teaching assistant (approx. 80 hours).
The course covered both shared-memory and message-passing
concurrency, using Java and Erlang as reference languages.
I provided assistance in lab sessions, grading of assignments and exams, and
Erlang coding tutorials.
%
% \cvlistitem{Concurrent systems. Communicating automata. Behavioural equivalences.}
% \cvlistitem{Real-time systems. Timed automata.}
% \cvlistitem{Process algebras: CCS, LNT.}
% \cvlistitem{Modal and temporal logics: HML, $\mu$-calculus, MCL. Model checking.}
% \cvlistitem{Model-based testing. Input-output conformance. Test case generation.}%
}
\ifShort{%
\begin{tightemize}
\item[] Shared-memory (Java) and message-passing (Erlang) concurrency
\item[] Approx. 80 hours of teaching assistance (lab sessions, grading, Erlang tutorial)
}
}

\comboJobZero
{Polytech Paris-Saclay}
{MSc course in Modelling and Verification}
{\MONTH{3} – \MONTH{4} 2022}
{Paris, \FR}
{}
\comboJob
{}
{}
{\MONTH{4} – \MONTH{5} 2021}
{}
{
\ifLong{%
% \cvline{
36-hour course for Master students in Computer Science Engineering,
\textit{filière apprentissage}.
Held remotely (2021) and in person (2022) as a supply teacher (\emph{intervenant vacataire}).
The course focused on the following topics:
modelling of concurrent systems through communicating automata (labelled
transition systems); behavioural equivalences; process algebras
(CCS, LNT); modal and temporal logics (HML, $\mu$-calculus, MCL);
model checking; modelling of real-time systems through timed automata;
model-based testing.
Tools such as CADP and UPPAAL were showcased in lab sessions.
%
%
% \cvlistitem{Concurrent systems. Communicating automata. Behavioural equivalences.}
% \cvlistitem{Real-time systems. Timed automata.}
% \cvlistitem{Process algebras: CCS, LNT.}
% \cvlistitem{Modal and temporal logics: HML, $\mu$-calculus, MCL. Model checking.}
% \cvlistitem{Model-based testing. Input-output conformance. Test case generation.}%
}
%
\ifShort{%
\begin{tightemize}
    \item[] 36-hour course for Master students in Computer Science Engineering
    (\textit{filière apprentissage}). Held remotely as a supply teacher (\emph{intervenant vacataire}).
    \item Concurrent, reactive, real-time systems. Communicating automata. Timed automata. 
    \item Behavioral equivalences of communicating automata.
    \item Process algebras: CCS, LNT. Modal and temporal logics: HML, $\mu$-calculus, MCL.
    \item Model-based testing. Input-output conformance. Test case generation.
    \item Lab sessions: 
    Uppaal (verification of timed automata);
    CADP (verification of concurrent systems);
    Testor (test case generation).
\end{tightemize}%
}
}



\section*{Other academic activities}

\subsection{Reviewing activity}

Invited as reviewer or sub-reviewer for the following conferences:

\cvlistitem{International Conference on Autonomous Agents and Multiagent Systems (AAMAS)}
\cvlistitem{International Symposium on Formal Methods (FM)}
\cvlistitem{International Conference on Software Technologies (ICSOFT)}
\cvlistitem{International Conference on Software Engineering and Formal Methods (SEFM)}
\cvlistitem{Theoretical Aspects of Software Engineering Conference (TASE)}

\subsection*{Participation to PhD schools}

\comboJob
{1st VMCAI Winter School}
{\ifShort{Attendee}}
{\MONTH{1} 2019}
{Lisbon, \PT}
{\ifLong{
\begin{tightemize}
    \item Computing with SAT oracles;
    \item Abstract interpretation;
    \item Modeling and verification of distributed protocols.
\end{tightemize}
}}

\section*{Publications}
\nocite{*}
\printbibliography[heading=journals, check=isjournal]
\printbibliography[heading=conferences, check=isconference]

\subsection{Citation indices}

Accessed on Scopus on \today.

\cvline{Citations}{15}
\cvline{\emph{h}-index}{3}  

\end{document}